\begin{rSection}{Other Projects}

\begin{rSubsection}{Global Undergraduate Awards}{September 2021}{Dr Ned Nedialkov}{Fully Automated Jigsaw Puzzle Solving by Hybrid ML}
	\begin{itemize}
      \addtolength\itemsep{-0.5em}
      \item Won first place in North America for a paper on modular CNN with random forest classification technique to solve square-piece jigsaws.
      \item Reported state-of-the-art piece-wise matching accuracy (around 98\% on 16$\times$16 pixel piece pairs and 99\% on 32$\times$32).
    \end{itemize}
\end{rSubsection}

%Matrix factorization
\begin{rSubsection}{National Big Data Competition}{June 2020}{Dr Yasaman Amannejad}{Medication Recommendation by Matrix Factorization}
	\begin{itemize}
      \addtolength\itemsep{-0.5em}
      \item Devised a matrix factorization-based recommender system to predict effective drugs for treating several mental illnesses, given a patient's history with other drugs.
      \item Performed data-scraping of approximately 50,000 records from Drugs.com.
    \end{itemize}
\end{rSubsection}

%Agent-based modelling
%\begin{rSubsection}{Coursework}{April 2018}{}{Agent-Based Modelling to Simulate Tumour Growth and Progression}
%	Simulated canine transmissible venereal tumours and the effects of the body's immunohistological environment on the tumour, specifically regarding MHC expression and Ig concentration.
%\end{rSubsection}

%Prime Numbers 
%\begin{rSubsection}{Enrichment Project}{January 2019 - April 2019}{Dr George Dragomir}{Prime Distribution by Linear Flow on the Torus}
%	By distributing the natural numbers over the surface of a torus by irrational linear flow, it was observed that primes could be partitioned by their residue class - and, more interestingly, distinctly from the direction of flow.
%\end{rSubsection}


%{\bf Predicting Drug-Drug Interactions in the Body using Minimal-Input Neural Networks}\\
%This was a data science competition focusing on the effects of recreational drugs. We designed a neural network that would predict, using only experimental properties of a compound, and with no knowledge of drug structure, whether or not two compounds would interact in the body. This achieved a state-of-the-art accuracy of 94.2\%. We presented a seminar on this paper at York University, and the paper has been published in the STEM Fellowship Journal. \\

%{\bf Using Agent-Based Modelling to Simulate Tumour Growth and Progression}\\
%Using agent-based modelling, canine transmissible venereal tumours were simulated, and the effects of various treatment methods on this growth were examined. Specifically, we investigated the immunohistological environment of the tumour and how changing MHC expression and various Ig concentrations affected tumour spread and virility. \\

% {\bf An Experiment in Plant-Animal Interactions}\\
% In this project, we designed and carried out a factorial experiment to assess the type and extent of herbivory by \textit{Myzus persicae} on \textit{Arabadopsis thaliana}. Specifically, we examined the effect that soil-nitrate content had on this relationship. A paper was written discussing the findings and statistical analyses applied, after which a small presentation was given to summarize the results. \\

\end{rSection}