\begin{rSection}{Employment}


%Statcan (Old)
%\begin{rSubsectionBullets}{Data Scientist}{June 2020 - August 2020}{Statistics Canada }{Consumer Prices Division}
%\item Developed natural language ML methods for hierarchical data structure mapping to aid in calculating the consumer price index.
%\end{rSubsectionBullets}

\begin{rSubsection}{Graduate Research and Teaching Assistant (Dessins d'Enfants)}{September 2021 - April 2023}{McMaster University}{Dr Cameron Franc}
	\begin{itemize}
      \addtolength\itemsep{-0.5em}
      %\item Continuing my theoretical work on dessins d'enfants, both in terms\\ homotopy continuation schemes and theoretical analysis of certain \\
      %    infinite families (I unfortunately should not just yet describe pre-publication)
      \item Continuing my theoretical work on dessins d'enfants, both in terms homotopy continuation schemes and theoretical analysis of certain infinite families of dessins induced by group actions
%      \item Working as a teaching assistant in at least 2 courses per semester at both the graduate and undergraduate level.
    \end{itemize}
\end{rSubsection}

%USRA
\begin{rSubsection}{Research Assistant (Number Theory and Symbolic ML)}{May 2021 - August 2021}{McMaster University}{Dr Cameron Franc}
	\begin{itemize}
       \addtolength\itemsep{-0.5em}
%      \item Investigated machine learning strategies to discriminate non-congruence finite-index subgroups of the modular group and compute Belyi maps corresponding to dessins d'enfants.
      \item Designed a symbolic evolutionary learning framework to allow the machine learning  of discrete number theoretic and algebraic problems from statistical and probabilistic data ---to the end of identifying congruential subgroup membership.
    \end{itemize}
\end{rSubsection}

%Statcan
\begin{rSubsection}{Data Scientist (NLP and the CPI)}{June 2020 - August 2020}{Statistics Canada }{Consumer Prices Division (Serge Goussev)}
    \begin{itemize}
      \addtolength\itemsep{-0.5em}
      \item Employed numerous NLP methods (including novel strategies) for hierarchical data structure mapping between non-isomorphic trees (representing store inventories) to more quickly calculate the consumer price index.
%      \item Included data manipulation and cleaning before use, and exploratory data techniques to determine appropriate methods.
      \item Learned to quickly write meaningful literature reviews on then current state-of-the-art methods, and then about the new state-of-the-art I achieved as I wrote technical reports on  my work.
    \end{itemize}
\end{rSubsection}

%Research assistant (Stewart Award) (Old)
%\begin{rSubsectionBullets}{Research Assistant}{May 2020 - July 2020}{McMaster University}{Dr George Dragomir, Dr Andy Nicas}
%\item Built on work by Dragomir and Nicas on quasi-hyperbolicity of metric spaces
%\item Investigated networks where quasi-hyperbolicity could be exploited to reduce embedding distortion
%\end{rSubsectionBullets}

%Research assistant (Stewart Award)
\begin{rSubsection}{Research Assistant (Quasi-Hyperbolicity and GNNs)}{May 2020 - July 2020}{McMaster University}{Drs George Dragomir and Andy Nicas}
    \begin{itemize}
      \addtolength\itemsep{-0.5em}
       \item Finite metric spaces (here, graphs) are generically not precisely any of the usual topological shapes we study -- and so we study how much they deviate from exactness.
       \item Doing so combinatorially, a graph on $<5000$ vertices would naively take longer to compute than the sun will exist -- with even `fast' methods having $\mathcal{O}(n^4)$ time-complexity.
%      \item Built on recent work to investigate how quasi-hyperbolicity could be exploited to reduce roughness and distortion in graph embeddings.
      \item Using particularly designed GCN-based models, I achieved unprecedented (and previously unseen) accuracy at predicting this hyperbolicity in constant time --- with the most fascinating contribution being my novel node features generation that appears to allow encoded global structure at the level of node-groups.
    \end{itemize}
\end{rSubsection}


%Research assistant (Breast scans) (Old)
%\begin{rSubsectionBullets}{Research Assistant}{May 2019 - May 2020}{McMaster University}{Dr Ned Nedialkov}
%%\item Developed convolutional neural networks to segment photoacoustic and ultrasound breast cancer scans
%\item	Worked parallel with a group from Western University to develop convolutional neural networks for segmentation for use in a hand-held \textit{in-situ} photoacoustic scanner
%%\item Achieved state-of-the-art accuracy at this task
%\end{rSubsectionBullets}

%Research assistant (Breast scans)
\begin{rSubsection}{Research Assistant (CNNs for Biomedical Applications)}{May 2019 - May 2020}{McMaster University}{Dr Ned Nedialkov}
  \begin{itemize}
    \addtolength\itemsep{-0.5em}
    \item Developed novel convolutional neural networks to segment photoacoustic cancerous breast tissue images.
    \item Used sophisticated techniques (eg, autoencoder-preprocessing over the Fourier-domain) to mitigate the unique style of photoacoustic noise rarely seen in other medical imaging.
    \item Developed data pipelines and infrastructure with an automated experiment tracking, ranking, monitoring, and batching software to train 1000s of models simultaneously for aggressive (given my access to at maximum 4000 GPUs simultaneously) hyperparameter optimisation.
    \item Networks used for intrasurgical device to assess tumour boundary \emph{during} operations without a radiologist, to reduce reoccurrence rate.
  \end{itemize}
\end{rSubsection}

%Tutoring (Old)
%\begin{rSubsectionBullets}{Math \& Computer Science Tutor}{December 2013 - Present}{Private}{}
%\item Worked one-on-one with each of two students to develop skills in math and computer programming.
%\end{rSubsectionBullets}

%Tutoring
%\begin{rSubsection}{Math \& Computer Science Tutor}{December 2013 - Present}{Private}{}
%	Worked one-on-one with each of two students to develop skills in math and computer programming.
%\end{rSubsection}

\end{rSection}