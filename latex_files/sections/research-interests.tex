\begin{rSection}{Research Interests and Profile}

%Very recent interests include homotopy continuation and symbolic regression.
%I am interested in additive number theory, Diophantine equations, and prime numbers.
%I am currently interested in dessins d'enfants and the computation of their Belyi maps from their permutation representations and passports. I write high-performance and distributed computing software, along with reasonably extensive experience with (pretty much the gamut of) machine learning techniques.

%Currently, I study \textbf{dessins d'enfants} and (often) the computation of \textbf{Belyi maps} from permutation representations and passports. For this, I write software (in all of \textbf{Julia, Python, and CUDA}) for \textbf{high-performance, parallel, and distributed computing}. Perhaps what most people are interested in is my extensive experience with a broad range of \textbf{machine learning (ML)} methods, particularly with \textbf{modern ML} techniques -- including \textbf{natural-language processing (NLP), computer vision (CV), neural networks} (including adversarial and generative networks), and \textbf{graph neural networks (GNNs)}. What I bring to the table beyond many data scientists is the rigorous mathematical training in \textbf{$\sigma$-algebras, Borel spaces, measure theory, topology}, and the like -- which gives me the fundamental understanding to perhaps work more meaningfully with these models than I otherwise may.

%My interests lie in {\bf mathematics and computing}, particularly {\bf number theory} and {\bf symbolic algebra}. Currently, I study {\bf dessins d'enfants} and (often) the computation of {\bf Belyi maps} from permutation representations and passports. The interested physicist will note that, outside the pure mathematics, these play a fundamental role in the search for appropriate models for String Theory. For this, I write software products (here, in all of {\bf Julia, Python, and CUDA}) for {\bf high-performance, parallel, and distributed computing}. Perhaps what most people are interested in is my experience across a broad range of {\bf machine learning (ML)} methods, particularly with modern ML techniques -- including {\bf natural-language processing (NLP), computer vision (CV)}, and some less standard neural networks including {\bf adversarial and generative networks}. Most recently, my work has focused on feature engineering for fundamentally featureless nodes when using {\bf graph neural networks (GNNs)}. What I bring to the table beyond many data scientists is the rigorous mathematical training in ideas such as {\bf $\sigma$-algebras, Borel spaces, measure theory, topology}, and the like -- all of which give me some fundamental understandings to work more meaningfully with these models -- in particular to iterate upon the fundamental computations more productively and in a direct manner rather than perhaps otherwise.

I currently study {\bf number theory} at McMaster University. My research primarily concerns two topics: the 4-category equivalence of which most people know the {\bf dessin d'enfant}, and the expressibility of the double parameterisation of discrete metric spaces by their {\bf additive} and (this is the novel bit) {\bf multiplicative hyperbolicity}.

The first topic, supposing it has an answer to the question of the action of the absolute Galois group, would change the landscape of arithmetic as we understand it, full stop. My work is both {\bf computational and theoretical in this domain}.
The second, as present, is just kind of interesting. However, how we tell whether we can or have found the ideal embedding of a metric space so as to analytically say anything meaningful is a problem that then becomes {\bf very interesting}, especially in the context of our parameterisation. Recently, I've devised significant, yet minimal node features for {\bf graph neural networks} to predict generally intractable {\bf routing approximation problems} by means of this hyperbolicity concept.

My specialty is in addressing these {\bf traditionally analytic, combinatorial, and algebraic problems} not only as they come, but by {\bf using methods in numerical mathematics} to solve otherwise intractable problems, and then {\bf regenerating exact solutions from approximations}. I also do a shocking amount of {\bf data analytics and machine learning} given my formal study, and all my research positions have ended up involving ML significantly.

%On the whole, my interests and skills are fairly eclectic, although it is number theory holds a special place in my heart. Despite this, I have had a wealth of experience with computing through various projects and classes, previous work, and my own research -- oftentimes the latter two having required me to learn new languages, skills, and even fields of mathematics of my own accord. I see this auto-didacticism as a skill that I can bring to almost any role -- and I am confident that this makes me a strong candidate for this position.
\end{rSection}