\begin{rSection}{Research Interests}

%Very recent interests include homotopy continuation and symbolic regression.
%I am interested in additive number theory, Diophantine equations, and prime numbers.
%I am currently interested in dessins d'enfants and the computation of their Belyi maps from their permutation representations and passports. I write high-performance and distributed computing software, along with reasonably extensive experience with (pretty much the gamut of) machine learning techniques.

%Currently, I study \textbf{dessins d'enfants} and (often) the computation of \textbf{Belyi maps} from permutation representations and passports. For this, I write software (in all of \textbf{Julia, Python, and CUDA}) for \textbf{high-performance, parallel, and distributed computing}. Perhaps what most people are interested in is my extensive experience with a broad range of \textbf{machine learning (ML)} methods, particularly with \textbf{modern ML} techniques -- including \textbf{natural-language processing (NLP), computer vision (CV), neural networks} (including adversarial and generative networks), and \textbf{graph neural networks (GNNs)}. What I bring to the table beyond many data scientists is the rigorous mathematical training in \textbf{$\sigma$-algebras, Borel spaces, measure theory, topology}, and the like -- which gives me the fundamental understanding to perhaps work more meaningfully with these models than I otherwise may.

My interests lie in \textbf{mathematics and computing}, particularly \textbf{number theory} and \textbf{symbolic algebra}. Currently, I study \textbf{dessins d'enfants} and (often) the computation of \textbf{Belyi maps} from permutation representations and passports. For this, I write software products (here, in all of \textbf{Julia, Python, and CUDA}) for \textbf{high-performance, parallel, and distributed computing}. Perhaps what most people are interested in is my experience across a broad range of \textbf{machine learning (ML)} methods, particularly with \textbf{modern ML} techniques -- including \textbf{natural-language processing (NLP), computer vision (CV), some more standard neural networks} (including adversarial and generative networks), and \textbf{graph neural networks (GNNs)}. What I bring to the table beyond many data scientists is the rigorous mathematical training in ideas such as \textbf{$\sigma$-algebras, Borel spaces, measure theory, topology}, and the like -- all of which give me the fundamental understanding to work more meaningfully with these models as a data scientist.

%On the whole, my interests and skills are fairly eclectic, although it is number theory holds a special place in my heart. Despite this, I have had a wealth of experience with computing through various projects and classes, previous work, and my own research -- oftentimes the latter two having required me to learn new languages, skills, and even fields of mathematics of my own accord. I see this auto-didacticism as a skill that I can bring to almost any role -- and I am confident that this makes me a strong candidate for this position.
\end{rSection}