\begin{rSection}{Invited Talks and Seminars}

%Belyi maps presentation
\begin{rSubsection}{Algebra and Algebraic Geometry Seminar}{November 2021}{McMaster University}{An Introduction to Belyi Maps}
    \begin{itemize}
      \addtolength\itemsep{-0.5em}
      \item Gave a 30-minute presentation on dessins d'enfants, their relevance, and pertinent computational techniques used in my research open to McMaster's math faculty and graduate students.
    \end{itemize}
\end{rSubsection}
\smallskip

%CANDEV 
\begin{rSubsection}{CANDEV}{January 2020}{Government of Canada}{Transformer Embeddings to  Identify Course Redundancies}
    \begin{itemize}
      \addtolength\itemsep{-0.5em}
      \item Gave a short talk on our use of transfer-learning with a transformer model to cluster courses offered by the Canadian School of Public Service and identify redundancies in course offerings.
      \item Received several offers to interview given the quality of our work (led to StatCan job!)
    \end{itemize}
\end{rSubsection}
\smallskip

%BDC 
\begin{rSubsection}{Undergraduate Big Data Competition}{July 2019}{STEM Fellowship}{Predicting \textit{in-vivo} Drug Interactions Without Drug Structure}
	%\smallskip
	\begin{itemize}
      \addtolength\itemsep{-0.5em}
      \item A talk given with coauthors on our ML model for predicting \textit{in-vivo} drug-drug interactions using only analytical chemical properties (which was \emph{not} in the literature at the time).
      \item Used around 1.2 million drug interactions for model training.
      \item Conference held at York University.
    \end{itemize}
\end{rSubsection}

%\vspace{4.5em}


%{\bf Predicting Drug-Drug Interactions in the Body using Minimal-Input Neural Networks}\\
%This was a data science competition focusing on the effects of recreational drugs. We designed a neural network that would predict, using only experimental properties of a compound, and with no knowledge of drug structure, whether or not two compounds would interact in the body. This achieved a state-of-the-art accuracy of 94.2\%. We presented a seminar on this paper at York University, and the paper has been published in the STEM Fellowship Journal. \\

%{\bf Using Agent-Based Modelling to Simulate Tumour Growth and Progression}\\
%Using agent-based modelling, canine transmissible venereal tumours were simulated, and the effects of various treatment methods on this growth were examined. Specifically, we investigated the immunohistological environment of the tumour and how changing MHC expression and various Ig concentrations affected tumour spread and virility. \\

% {\bf An Experiment in Plant-Animal Interactions}\\
% In this project, we designed and carried out a factorial experiment to assess the type and extent of herbivory by \textit{Myzus persicae} on \textit{Arabadopsis thaliana}. Specifically, we examined the effect that soil-nitrate content had on this relationship. A paper was written discussing the findings and statistical analyses applied, after which a small presentation was given to summarize the results. \\

\end{rSection}